\documentclass{article}
\usepackage[T1]{fontenc}
\usepackage[utf8]{inputenc}

\usepackage{mathtools, amssymb, amsthm}
\usepackage{parskip}
\usepackage{xcolor}
\usepackage{pgfplotstable}
\pgfplotsset{compat=1.18}
\usepackage{float}
\usepackage{caption}
\usepackage{todonotes}
\usepackage{siunitx}
\usepackage{microtype}
\usepackage[english]{babel}
\usepackage[style=numeric, sorting=none]{biblatex}
\addbibresource{ummrc.bib}

\usepackage{bm}
\usepackage{booktabs}
\usepackage{hyperref}
\hypersetup{
  pdftitle={Mathematics and Statistics Research Competition Topic S–01 Integer Construction},
  pdfauthor={Edward Wang},
  colorlinks,
  linkcolor={red!50!black},
  citecolor={green!50!black},
  urlcolor={blue!80!black}
}
\usepackage{geometry}
\geometry{
  paperwidth=210mm,
  paperheight=297mm,
  textwidth=0.72\paperwidth,
  marginparwidth=2cm,
}
\usepackage[p]{ETbb}
\usepackage[libertine, vvarbb]{newtxmath}
\usepackage[lining, scale = 0.82]{FiraMono}
\usepackage{minted}
\setminted{fontsize = \normalsize, linenos, style = staroffice, breaklines}
\usepackage{csquotes}

\definecolor{cmap1}{HTML}{440154}
\definecolor{cmap2}{HTML}{443983}
\definecolor{cmap3}{HTML}{31688e}
\definecolor{cmap4}{HTML}{21918c}
\definecolor{cmap5}{HTML}{35b779}
\definecolor{cmap6}{HTML}{90d743}

\newtheorem{theorem}{Theorem}[section]
\newtheorem{conj}{Conjecture}[section]
\newtheorem{claim}{Claim}[section]
\newtheorem{lemma}{Lemma}[section]
\theoremstyle{definition}
\newtheorem{definition}{Definition}[section]
\newtheorem*{problem}{Problem}

\DeclarePairedDelimiter\abs{\lvert}{\rvert}
\DeclarePairedDelimiter\floor{\lfloor}{\rfloor}
\DeclarePairedDelimiter\ceil{\lceil}{\rceil}

\usetikzlibrary{automata, arrows, positioning, calc, graphs, graphs.standard}

\title{\bfseries Mathematics and Statistics \\ Research Competition \\ Topic S-01 Integer Construction}
\author{Jiamu Li \& Frank Tang \& Edward Wang \\[1em] Scotch College}
\date{\today}

\newcommand\NTen{\ensuremath{\mathcal{N}_{10, 2}}}
\newcommand\LTen{\ensuremath{\mathcal{L}_{10, 2}}}
\newcommand\NN{\ensuremath{\mathcal{N}}}
\newcommand\LL{\ensuremath{\mathcal{L}}}


\begin{document}
\maketitle
\tableofcontents
\clearpage

\newcommand\NTen{\ensuremath{\mathcal{N}_{10, 2}}}

Let \NTen{} be the set of positive integers whose digits in base-$10$ comprise only 0s and 1s. Examples of elements in \NTen{} are: $1001$, $110$, and $11$. Examples of elements not in \NTen{} are: $4201$, $690$, and $12$.

Consider a positive integer $N$. It can be constructed as the sum of elements in \NTen{}. For example, one construction of $1337$ with $8$ summands which are elements in \NTen{} is as follows
\[
  1337 = 1000 + 111 + 111 + 111 + 1 + 1 + 1 + 1
.\] 

\section{Problem 1}
\begin{problem}
  What are all of the constructions of $1337$ using elements of \NTen{}?
\end{problem}
We interpret the question as asking for how many unique ways there are to obtain $1337$ as a sum of elements from \NTen{}. To do this, we look at the general case which seeks to find the number of unique ways to obtain a positive integer  $n$ as a sum of elements from \NTen{}.
\begin{definition}
  For sake of convenience, we define a function $C(n)$ that counts the number of unique ways of constructing $n$ as a sum of elements in \NTen{}. That is, $C(n)$ is the number of unique constructions such that \[
    n = \sum_{a_i \in \NTen} a_i
  .\] 
\end{definition}
We start with an elementary example. Consider $n = 15$, and suppose we wish to find $C(15)$. Clearly, the only elements of \NTen{} that are relevant here are $1$, $10$ and $11$. With so few elements, we can easily calculate $C(15)$ manually. We find that there are three unique constructions of $15$, which are 
\begin{align*}
  15 &= 1 + 1 + 1 + 1 + 1 + 1 + 1 + 1 + 1 + 1 + 1 + 1 + 1 + 1 + 1 \\
  15 &= 1 + 1 + 1 + 1 + 1 + 10 \\
  15 &= 1 + 1 + 1 + 1 + 11.
\end{align*}
Hence $C(15) = 3$. However, this method quickly breaks down for large $n$, where the number of relevant elements of \NTen{} increases with the length of the number. For example, there are $15$ relevant elements of $\NTen$ for $n = 1337$, which are $1, 10, 11, 100, 101, 110, 111, 1000, 1001, 1010, 1011, 1100, 1101, 1110, 1111$. As such, we need a better way to count $C(n)$.

Let us consider a simpler case. Consider a fictional country Numberland whose currency system consists of \NTen. Suppose Dave wants to count the number of ways to make  $\$ 15$ in this currency system. This is equivalent to asking what $C(15)$ is.

Suppose the number of ways to make $\$ p$ using only $\$ 1$ coins is $m$, and the number of ways to make $\$ q$ using only $\$ 10$ coins is $n$. Now, consider the terms $mx^{p}$ and $nx^{q}$. Due to the multiplication principle, the number of ways to make $\$ p+q$ using coins of either $\$ 1$ or $\$ 10$ is obviously $mn$. This is equivalent to looking at the coefficient of $p+q$ after multiplying $mx^{p}$ and $nx^{q}$.
Now consider a \emph{generating function} that is defined as the polynomial $f(x) = \sum_{i=1}^{\infty} a_i x^i$. Let the coefficients $a_i$ describe the number of ways to construct $\$ i$ using coins of value $\$ p$. Now let another generating function be defined as $g(x) = \sum_{i=1}^{\infty} b_i x^i$, where the coefficients $b_i$ represent the number of ways to construct $\$ i$ using coins of value $\$ q$. If we multiply these functions together, we get another polynomial \[
  h(x) = f(x)g(x) = (a_0 + a_1x + a_2x^2 + \dots)(b_0 + b_1x + b_2x^2 + \dots) = \sum_{i=1}^{\infty} c_ix^{i}
.\] Now, as before, each coefficient $c_i$ represents the number of ways to make $i$ using coins of value $\$ p$ or $\$ q$. Now, let us consider the generating functions of the different coin values of Numberland.

Consider the generating function $g_1$ of coins of value $\$ 1$. Clearly, we can make values of integer value $\$ k$ in exactly $1$ way, that is, $k = 1 + \dots + 1$. Thus the generating function is simply \[
  g_{1}(x) = 1 + x + x^2 + \dots
.\] This is clearly a geometric series, and therefore we may simplify $g_1$ to \[
  g_1(x) = \frac{1}{1-x}
.\] Next, consider the generating function $g_{10}$ of coins of value $\$ 10$. Note that we can only make amounts that have values of multiples of  $10$. Moreover, we can only make those amounts in exactly $1$ way, which is $10k = 10 + \dots + 10$ for positive integers $k$. Thus we have \[
g_{10}(x) = \bm{1} + 0x + 0x^2 + \dots + 0x^{9} + \bm{x^{10}} + 0x^{11} + \dots + 0x^{19} + \bm{x^{20}} + \dots = 1 + x^{10} + x^{20} + \dots
.\] Again, we can rewrite this as geometric series in the form \[
  g_{10}(x) = \frac{1}{1-x^{10}}
.\] One may now notice a pattern that we now prove more formally.
\begin{theorem}
  The generating function for coins of value $a$ is \[
    g_a(x) = \frac{1}{1-x^a}
  .\] 
\end{theorem}
\begin{proof}
  i really can't be fucked rn
\end{proof}
\begin{theorem}
  For positive integers $n$, $C(n)$ is equal to the coefficient of $x^{n}$ in \[
    \prod_{a_i \in \NTen} \frac{1}{(1-x^{a_i})}
  .\] 
\end{theorem}
\begin{proof}
  yeah well we kinda proved it above didn't we
\end{proof}

\section{Problem 2}
\begin{definition}
  Let $\mathcal{L}_{10, 2}(N)$ be the minimum number of summands needed to construct $N$ using elements in \NTen. For example, $\mathcal{L}_{10, 2}(13) = 3$.
\end{definition}
\begin{problem}
Determine $\mathcal{L}_{10, 2}(N)$ for the following values of $N$:
\begin{itemize}
  \item $1337$
  \item  $12345$
  \item  $190274876$
\end{itemize}
\end{problem}
\begin{definition}
Let $d_N$ be the set of digits in the base $10$ expansion of $N$.
\end{definition}
For example, $d_{1337} = \{1, 3, 7\}$.
\begin{theorem}
  For a positive integer $N$, $\mathcal{L}_{10, 2}(N) = \max\{d_N\}$.
\end{theorem}
\begin{theorem}
  For a positive integer $N$, we have \[\mathcal{L}_{10, k}(N) = \ceil*{\frac{\max\{d_N\}}{k-1}}.\]
\end{theorem}
\end{document}
