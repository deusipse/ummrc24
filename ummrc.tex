\documentclass{article}
\usepackage[T1]{fontenc}
\usepackage[utf8]{inputenc}

\usepackage{mathtools, amssymb, amsthm}
\usepackage{parskip}
\usepackage{xcolor}
\usepackage{pgfplotstable}
\pgfplotsset{compat=1.18}
\usepackage{float}
\usepackage{caption}
\usepackage{todonotes}
\usepackage{siunitx}
\usepackage{microtype}
\usepackage[english]{babel}
\usepackage[style=numeric, sorting=none]{biblatex}
\addbibresource{ummrc.bib}

\usepackage{bm}
\usepackage{booktabs}
\usepackage{hyperref}
\hypersetup{
  pdftitle={Mathematics and Statistics Research Competition Topic S–01 Integer Construction},
  pdfauthor={Edward Wang},
  colorlinks,
  linkcolor={red!50!black},
  citecolor={green!50!black},
  urlcolor={blue!80!black}
}
\usepackage{geometry}
\geometry{
  paperwidth=210mm,
  paperheight=297mm,
  textwidth=0.72\paperwidth,
  marginparwidth=2cm,
}
\usepackage[p]{ETbb}
\usepackage[libertine, vvarbb]{newtxmath}
\usepackage[lining, scale = 0.82]{FiraMono}
\usepackage{minted}
\setminted{fontsize = \normalsize, linenos, style = staroffice, breaklines}
\usepackage{csquotes}

\definecolor{cmap1}{HTML}{440154}
\definecolor{cmap2}{HTML}{443983}
\definecolor{cmap3}{HTML}{31688e}
\definecolor{cmap4}{HTML}{21918c}
\definecolor{cmap5}{HTML}{35b779}
\definecolor{cmap6}{HTML}{90d743}

\newtheorem{theorem}{Theorem}[section]
\newtheorem{conj}{Conjecture}[section]
\newtheorem{claim}{Claim}[section]
\newtheorem{lemma}{Lemma}[section]
\theoremstyle{definition}
\newtheorem{definition}{Definition}[section]
\newtheorem*{problem}{Problem}

\DeclarePairedDelimiter\abs{\lvert}{\rvert}
\DeclarePairedDelimiter\floor{\lfloor}{\rfloor}
\DeclarePairedDelimiter\ceil{\lceil}{\rceil}

\usetikzlibrary{automata, arrows, positioning, calc, graphs, graphs.standard}

\title{\bfseries Mathematics and Statistics \\ Research Competition \\ Topic S-01 Integer Construction}
\author{Jiamu Li \& Frank Tang \& Edward Wang \\[1em] Scotch College}
\date{\today}

\newcommand\NTen{\ensuremath{\mathcal{N}_{10, 2}}}
\newcommand\LTen{\ensuremath{\mathcal{L}_{10, 2}}}
\newcommand\NN{\ensuremath{\mathcal{N}}}
\newcommand\LL{\ensuremath{\mathcal{L}}}


\begin{document}
\maketitle
\tableofcontents
\clearpage

\newcommand\NTen{\ensuremath{\mathcal{N}_{10, 2}}}

Let \NTen{} be the set of positive integers whose digits in base-$10$ comprise only 0s and 1s. Examples of elements in \NTen{} are: $1001$, $110$, and $11$. Examples of elements not in \NTen{} are: $4201$, $690$, and $12$.

Consider a positive integer $N$. It can be constructed as the sum of elements in \NTen{}. For example, one construction of $1337$ with $8$ summands which are elements in \NTen{} is as follows
\[
  1337 = 1000 + 111 + 111 + 111 + 1 + 1 + 1 + 1
.\] 

\section{Problem 1}
\begin{problem}
  What are all of the constructions of $1337$ using elements of \NTen{}?
\end{problem}
We interpret the question as asking for how many unique ways there are to obtain $1337$ as a sum of elements from \NTen{}. To do this, we look at the general case which seeks to find the number of unique ways to obtain a positive integer  $n$ as a sum of elements from \NTen{}.
\begin{definition}
  For sake of convenience, we define a function $C(n)$ that counts the number of unique ways of constructing $n$ as a sum of elements in \NTen{}. That is, $C(n)$ is the number of unique constructions such that \[
    n = \sum_{a_i \in \NTen} a_i
  .\] 
\end{definition}
We start with an elementary example. Consider $n = 15$, and suppose we wish to find $C(15)$. 

  
\end{document}
