\documentclass{article}
\usepackage[T1]{fontenc}
\usepackage[utf8]{inputenc}

\usepackage{mathtools, amssymb, amsthm}
\usepackage{parskip}
\usepackage{xcolor}
\usepackage{pgfplotstable}
\pgfplotsset{compat=1.18}
\usepackage{float}
\usepackage{caption}
\usepackage{todonotes}
\usepackage{siunitx}
\usepackage{microtype}
\usepackage[english]{babel}
\usepackage[style=numeric, sorting=none]{biblatex}
\addbibresource{ummrc.bib}

\usepackage{bm}
\usepackage{booktabs}
\usepackage{hyperref}
\hypersetup{
  pdftitle={Mathematics and Statistics Research Competition Topic S–01 Integer Construction},
  pdfauthor={Edward Wang},
  colorlinks,
  linkcolor={red!50!black},
  citecolor={green!50!black},
  urlcolor={blue!80!black}
}
\usepackage{geometry}
\geometry{
  paperwidth=210mm,
  paperheight=297mm,
  textwidth=0.72\paperwidth,
  marginparwidth=2cm,
}
\usepackage[p]{ETbb}
\usepackage[libertine, vvarbb]{newtxmath}
\usepackage[lining, scale = 0.82]{FiraMono}
\usepackage{minted}
\setminted{fontsize = \normalsize, linenos, style = staroffice, breaklines}
\usepackage{csquotes}

\definecolor{cmap1}{HTML}{440154}
\definecolor{cmap2}{HTML}{443983}
\definecolor{cmap3}{HTML}{31688e}
\definecolor{cmap4}{HTML}{21918c}
\definecolor{cmap5}{HTML}{35b779}
\definecolor{cmap6}{HTML}{90d743}

\newtheorem{theorem}{Theorem}[section]
\newtheorem{conj}{Conjecture}[section]
\newtheorem{claim}{Claim}[section]
\newtheorem{lemma}{Lemma}[section]
\theoremstyle{definition}
\newtheorem{definition}{Definition}[section]
\newtheorem*{problem}{Problem}

\DeclarePairedDelimiter\abs{\lvert}{\rvert}
\DeclarePairedDelimiter\floor{\lfloor}{\rfloor}
\DeclarePairedDelimiter\ceil{\lceil}{\rceil}

\usetikzlibrary{automata, arrows, positioning, calc, graphs, graphs.standard}

\title{\bfseries Mathematics and Statistics \\ Research Competition \\ Topic S-01 Integer Construction}
\author{Jiamu Li \& Frank Tang \& Edward Wang \\[1em] Scotch College}
\date{\today}

\newcommand\NTen{\ensuremath{\mathcal{N}_{10, 2}}}
\newcommand\LTen{\ensuremath{\mathcal{L}_{10, 2}}}
\newcommand\NN{\ensuremath{\mathcal{N}}}
\newcommand\LL{\ensuremath{\mathcal{L}}}


\begin{document}
\maketitle
\tableofcontents
\clearpage

\newcommand\NTen{\ensuremath{\mathcal{N}_{10, 2}}}
\newcommand\LTen{\ensuremath{\mathcal{L}_{10, 2}}}

Let \NTen{} be the set of positive integers whose digits in base-$10$ comprise only 0s and 1s. Examples of elements in \NTen{} are: $1001$, $110$, and $11$. Examples of elements not in \NTen{} are: $4201$, $690$, and $12$.

Consider a positive integer $N$. It can be constructed as the sum of elements in \NTen{}. For example, one construction of $1337$ with $8$ summands which are elements in \NTen{} is as follows
\[
  1337 = 1000 + 111 + 111 + 111 + 1 + 1 + 1 + 1
.\] 

\section{Problem 1}
\begin{problem}
  What are all of the constructions of $1337$ using elements of \NTen{}?
\end{problem}
We interpret the question as asking for how many unique ways there are to obtain $1337$ as a sum of elements from \NTen{}. To do this, we look at the general case which seeks to find the number of unique ways to obtain a positive integer  $n$ as a sum of elements from \NTen{}.
\begin{definition}
  For sake of convenience, we define a function $C(n)$ that counts the number of unique ways of constructing $n$ as a sum of elements in \NTen{}. That is, $C(n)$ is the number of unique constructions such that \[
    n = \sum_{a_i \in \NTen} a_i
  .\] 
\end{definition}
We start with an elementary example. Consider $n = 15$, and suppose we wish to find $C(15)$. Clearly, the only elements of \NTen{} that are relevant here are $1$, $10$ and $11$. With so few elements, we can easily calculate $C(15)$ manually. We find that there are three unique constructions of $15$, which are 
\begin{align*}
  15 &= 1 + 1 + 1 + 1 + 1 + 1 + 1 + 1 + 1 + 1 + 1 + 1 + 1 + 1 + 1 \\
  15 &= 1 + 1 + 1 + 1 + 1 + 10 \\
  15 &= 1 + 1 + 1 + 1 + 11.
\end{align*}
Hence $C(15) = 3$. However, this method quickly breaks down for large $n$, where the number of relevant elements of \NTen{} increases with the length of the number. For example, there are $15$ relevant elements of $\NTen$ for $n = 1337$, which are $1, 10, 11, 100, 101, 110, 111, 1000, 1001, 1010, 1011, 1100, 1101, 1110, 1111$. As such, we need a better way to count $C(n)$.

Let us consider a simpler case. Consider a fictional country Numberland whose currency system consists of \NTen. Suppose Dave wants to count the number of ways to make  $\$ 15$ in this currency system. This is equivalent to asking what $C(15)$ is.

Suppose the number of ways to make $\$ p$ using only $\$ 1$ coins is $m$, and the number of ways to make $\$ q$ using only $\$ 10$ coins is $n$. Now, consider the terms $mx^{p}$ and $nx^{q}$. Due to the multiplication principle, the number of ways to make $\$ p+q$ using coins of either $\$ 1$ or $\$ 10$ is obviously $mn$. This is equivalent to looking at the coefficient of $p+q$ after multiplying $mx^{p}$ and $nx^{q}$.
Now consider a \emph{generating function} that is defined as the polynomial $f(x) = \sum_{i=1}^{\infty} a_i x^i$. Let the coefficients $a_i$ describe the number of ways to construct $\$ i$ using coins of value $\$ p$. Now let another generating function be defined as $g(x) = \sum_{i=1}^{\infty} b_i x^i$, where the coefficients $b_i$ represent the number of ways to construct $\$ i$ using coins of value $\$ q$. If we multiply these functions together, we get another polynomial \[
  h(x) = f(x)g(x) = (a_0 + a_1x + a_2x^2 + \dots)(b_0 + b_1x + b_2x^2 + \dots) = \sum_{i=1}^{\infty} c_ix^{i}
.\] Now, as before, each coefficient $c_i$ represents the number of ways to make $i$ using coins of value $\$ p$ or $\$ q$. Now, let us consider the generating functions of the different coin values of Numberland.

Consider the generating function $g_1$ of coins of value $\$ 1$. Clearly, we can make values of integer value $\$ k$ in exactly $1$ way, that is, $k = 1 + \dots + 1$. Thus the generating function is simply \[
  g_{1}(x) = 1 + x + x^2 + \dots
.\] This is clearly a geometric series, and therefore we may simplify $g_1$ to \[
  g_1(x) = \frac{1}{1-x}
.\] Next, consider the generating function $g_{10}$ of coins of value $\$ 10$. Note that we can only make amounts that have values of multiples of  $10$. Moreover, we can only make those amounts in exactly $1$ way, which is $10k = 10 + \dots + 10$ for positive integers $k$. Thus we have \[
g_{10}(x) = \bm{1} + 0x + 0x^2 + \dots + 0x^{9} + \bm{x^{10}} + 0x^{11} + \dots + 0x^{19} + \bm{x^{20}} + \dots = 1 + x^{10} + x^{20} + \dots
.\] Again, we can rewrite this as geometric series in the form \[
  g_{10}(x) = \frac{1}{1-x^{10}}
.\] One may now notice a pattern that we now prove more formally.
\begin{theorem}
  The generating function for coins of value $a$ is \[
    g_a(x) = \frac{1}{1-x^a}
  .\] 
\end{theorem}
\begin{proof}
  Given coins of value $a$, we can only make amounts of value $ka$, where $k$ is an integer. Moreover, we can make these amounts in exactly one way, with $k$ coins. Thus the generating function, which can be rewritten as a geometric series, must be \[
    g_a(x) = x^{0a} + x^{a} + x^{2a} + \dots = \frac{1}{1-x^{a}} \qedhere
  .\] 
\end{proof}
\begin{theorem}
  For positive integers $n$, $C(n)$ is equal to the coefficient of $x^{n}$ in \[
    \prod_{a_i \in \NTen} \frac{1}{(1-x^{a_i})}
  .\] 
\end{theorem}
\begin{proof}
  Recall that we can multiply generating functions together, and the resulting coefficients of each term $x^{n}$ represents the number of ways to form $n$. Thus we multiply the generating functions for all the numbers in the set \NTen to get the function \[
    \prod_{a_i\in \NTen} \frac{1}{(1-x^{a_i})}
  ,\] where the coefficient of $x^n$ denotes the number of ways to form $n$, which is equivalent to $C(n)$.
\end{proof}
Unfortunately, there exists no closed form expression to determine $C(n)$. %insert graph here please ed
However, by using computer software like Mathematica, we may calculate $C(n)$, to get $C(1337) = 1448684$. We may graph the function $C(n)$, and although it looks exponential, the function is subexponential if we use a logarithmic scale. This is commonly known as a restricted partition function, which aympototically approaches \[
  C(n) \sim \frac{a}{n}e^{b\sqrt{n}}
.\] % add a reference

\section{Problem 2}
\begin{definition}
  Let $\mathcal{L}_{10, 2}(N)$ be the minimum number of summands needed to construct $N$ using elements in \NTen.
\end{definition}
\begin{problem}\label{p2}
Determine $\mathcal{L}_{10, 2}(N)$ for the following values of $N$:
\begin{itemize}
  \item $1337$
  \item  $12345$
  \item  $190274876$
\end{itemize}
\end{problem}
For sake of convenience, we define a $d_{n, i}$ to be the $i$th digit of $n$.
\begin{definition}
  For an integer $n$, the $i$th digit of  $n$, with the leftmost digit being the first, is denoted $d_{n, i}$.
\end{definition}
For example, $d_{1337, 4} = 7$.
\begin{theorem}
  For a positive integer $N$, $\mathcal{L}_{10, 2}(N) = \max\{d_{n, i}\}$.
\end{theorem}
\begin{proof}
  Firstly, we recognise that $\LTen(n)$ is equivalent to the minimum number of subtractions to make $0$ from $n$, using elements of \NTen. Intuitively, it is obvious that an optimal strategy would be to subtract $1$ from every non-zero digit in $n$, which decreases the largest digit by $1$ every time. This means that the total number of subtractions is simply equal to the largest digit of $n$. We can prove that this is the optimal strategy as part of a more general theorem.
\end{proof}
\begin{theorem}
  For a positive integer $N$, we have \[\mathcal{L}_{10, k}(N) = \ceil*{\frac{\max\{d_{n, i}\}}{k-1}}.\]
\end{theorem}
\begin{proof}
  Recall that in \NTen{}, the largest available digit is $k-1$. Again, the optimal strategy is to subtract $k-1$ from the largest digit in $n$ at every subtraction, until the largest digit is less than $k-1$, at which point it is obvious that it takes one more subtraction to reach zero. This gives us a total number of \[
    \ceil*{\frac{\max\{d_{n, i\}}}{j-1}}
  \] subtractions. We must now prove that it is impossible to do better than this. 

  Firstly, notice that if we subtract using numbers that are less than the largest digit every time, then the largest digit decreases by less than $k-1 $ each subtraction, meaning that we cannot beat the previous minimum number of subtractions. However, if we use subtractions that use digits larger than the largest digit in $n$, such as in $123-4$, we must be more careful.

  Consider the representation of a 4 digit number $n$ as concatenated digits $a|b|c|d$. Note that we can `group' digits to make an equivalent representation $(10a + b)|c|d$, in other words, we will allow `digits' that have two numbers.

  Now consider the subtraction $123-4$. The usual way we subtract numbers is that we `carry' a $1$ to the $3$ to do the subtraction $1|1|13 - 4 = 119$. Thus we see that the representation $a|b|c|d$ is equivalent to the representation $a|b|(c-1)|(d+10)$. This only occurs when we subtract using a digit that is greater than the largest digit in $n$. Notice that if we do not use any `alternate' representations, with the digits in their usual place, then every digit will be between $0$ and $9$, inclusive. This means that the \emph{maximum} number of subtractions must be \[
    \ceil*{\frac{9}{k-1}}
  .\] However, note that if we subtract using digits larger than the largest digit in  $n$, then we must have some pattern of  $(c-1)|(d+10)$ in the representation of  $n$. This means that the largest digit of $n$ now ranges from $10$ to $19$, inclusive. Importantly, this means that the \emph{minimum} number of subtractions is now \[
  \ceil*{\frac{10}{k-1}} \ge \ceil*{\frac{9}{k-1}}
  .\] Thus it always takes more (or the same) number of subtractions if we subtract using digits that are larger than the largest digit in $n$. Hence, the optimal strategy of subtracting using digits that are equal to the largest digit, is indeed the best possible strategy.
\end{proof}
Using this, we can easily determine the answers to Problem~\ref{p2}, which are in Table~\ref{t1}.
\begin{table}[H]
  \centering
  \begin{tabular}{lc} \toprule
    $n$ & $\LTen(n)$ \\ \midrule
    1337 & 7 \\
    12345 & 5 \\
    190274876 & 9 \\ \bottomrule
  \end{tabular}
  \caption{}
  \label{t1}
\end{table}

\section{Problem 3}
In general, let  $\mathcal{N}_{10, k}$ be the set of positive integers whose digits in base-10 are strictly less than $k$, where $k$ is a given positive integer. Let $\mathcal{L}_{10,k}(N)$ be the minimum number of summands needed to construct $N$ using elements in $\mathcal{N}_{10,k}$.

\begin{problem}
  Given the definitions, 
  \begin{itemize}
    \item Explore the relationship between $k$ and $\mathcal{L}_{10, k}(N)$.
    \item Define $L_{j, k}(N)$ and investigate the nature of $L_{j, k}(N)$ for positive integers $j, k$.
  \end{itemize}
\end{problem}

\begin{figure}[htbp]
  \centering
  \begin{tikzpicture}
    \pgfplotstableread{graph/something.dat}{\decreasing}
    \begin{axis}[height = 6cm, width = 0.9\columnwidth, xlabel={Number of hives, $m$}, ylabel={Expected time, $E_1$}, title = {Expected time for all hives to be infected vs number of hives}]
      \addplot[thick, cmap1] table {\decreasing};
    \end{axis}
  \end{tikzpicture}
  \caption{The expected time decreases as $m$ increases, $p = 0.2$.}
  \label{fig:decreasing}
\end{figure}
\begin{figure}[htbp]
  \centering
  \begin{tikzpicture}
    \pgfplotstableread{graph/something2.dat}{\decreasing}
    \begin{axis}[height = 6cm, width = 0.9\columnwidth, xlabel={Number of hives, $m$}, ylabel={Expected time, $E_1$}, title = {Expected time for all hives to be infected vs number of hives}]
      \addplot[thick, cmap1] table {\decreasing};
    \end{axis}
  \end{tikzpicture}
  \caption{The expected time decreases as $m$ increases, $p = 0.2$.}
  \label{fig:decreasing2}
\end{figure}
\begin{figure}[htbp]
  \centering
  \begin{tikzpicture}
    \pgfplotstableread{graph/something2.dat}{\decreasing}
    \begin{axis}[height = 6cm, width = 0.9\columnwidth, xlabel={Number of hives, $m$}, ylabel={Expected time, $E_1$}, title = {Expected time for all hives to be infected vs number of hives}, ymode = log]
      \addplot[thick, cmap1] table {\decreasing};
    \end{axis}
  \end{tikzpicture}
  \caption{The expected time decreases as $m$ increases, $p = 0.2$.}
  \label{fig:decreasing3}
\end{figure}



\end{document}
