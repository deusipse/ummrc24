
Let \NTen{} be the set of positive integers whose digits in base-$10$ comprise only 0s and 1s. Examples of elements in \NTen{} are: $1001$, $110$, and $11$. Examples of elements not in \NTen{} are: $4201$, $690$, and $12$.

Consider a positive integer $N$. It can be constructed as the sum of elements in \NTen{}. For example, one construction of $1337$ with $8$ summands which are elements in \NTen{} is as follows
\[
  1337 = 1000 + 111 + 111 + 111 + 1 + 1 + 1 + 1
.\] 

\section{Problem 1}
\begin{problem}\label{p1}
  What are all of the constructions of $1337$ using elements of \NTen{}?
\end{problem}
We interpret the question as asking for how many unique ways there are to obtain $1337$ as a sum of elements from \NTen{}. To do this, we look at the general case which seeks to find the number of unique ways to obtain a positive integer  $n$ as a sum of elements from \NTen{}.
\begin{definition}
  For sake of convenience, we define a function $C(n)$ that counts the number of unique ways of constructing $n$ as a sum of elements in \NTen{}. That is, $C(n)$ is the number of unique constructions such that \[
    n = \sum_{a_i \in \NTen} a_i
  .\] 
\end{definition}
We start with an elementary example. Consider $n = 15$, and suppose we wish to find $C(15)$. Clearly, the only elements of \NTen{} that are relevant here are $1$, $10$ and $11$. With so few elements, we can easily calculate $C(15)$ manually. We find that there are three unique constructions of $15$, which are 
\begin{align*}
  15 &= 1 + 1 + 1 + 1 + 1 + 1 + 1 + 1 + 1 + 1 + 1 + 1 + 1 + 1 + 1 \\
  15 &= 1 + 1 + 1 + 1 + 1 + 10 \\
  15 &= 1 + 1 + 1 + 1 + 11.
\end{align*}
Hence $C(15) = 3$. However, this method quickly becomes impractical for large $n$, where the number of relevant elements of \NTen{} increases with the length of the number. For example, there are $15$ relevant elements of $\NTen$ for $n = 1337$, which are $1, 10, 11, 100, 101, 110, 111, 1000, 1001, 1010, 1011, 1100, 1101, 1110$ and $1111$. As such, we need a better way to count $C(n)$.

Let us consider a simpler case. Consider a fictional country Numberland whose currency system consists of \NTen. Suppose Dave wants to count the number of ways to make  $\$ 15$ in this currency system. This is equivalent to asking what $C(15)$ is.

% Suppose the number of ways to make $\$ p$ using only $\$ 1$ coins is $m$, and the number of ways to make $\$ q$ using only $\$ 10$ coins is $n$. Now, consider the terms $mx^{p}$ and $nx^{q}$. Due to the multiplication principle, the number of ways to make $\$ p+q$ using coins of either $\$ 1$ or $\$ 10$ is obviously $mn$. This is equivalent to looking at the coefficient of $p+q$ after multiplying $mx^{p}$ and $nx^{q}$.
Take a power series $f(x) = \sum_{i=0}^{\infty} a_ix^{i}$, whose coefficients $a_i$ denote the number of ways to make $\$ i$ using only $\$1$ coins. Clearly, \[
  f(x) = 1 + x + x^{2} + x^{3} + \dots
,\] because there is one way to make every integer using only $1$s. In addition, take the power series $g(x) = \sum_{i=0}^{\infty} b_ix^{i}$ whose coefficients $b_i$ represent the number of ways to make $\$ i$ using only $\$ 10$ coins. Since one can only make multiples of $\$10$ using $\$10$ coins, we have \[
g(x) = \bm{1} + 0x + 0x^2 + \dots + 0x^{9} + \bm{x^{10}} + 0x^{11} + \dots + 0x^{19} + \bm{x^{20}} + \dots = 1 + x^{10} + x^{20} + \dots
,\] since there is exactly one way to make every multiple of $10$. Now let us take the product $f(x)g(x)$ and find the coefficient of $x^{20}$. We have \[
  f(x)g(x) = (1+x+x^{2}+x^{3} + \dots) (1+x^{10} + x^{20} + x^{30} + \dots)
,\] and note that we can make $x^{20}$ from $1\times x^{20}$, $x^{10}\times x^{10}$ and $x^{20}\times 1$, meaning that the coefficient of $x^{20}$ is $3$. The key observation here is that each of the ways to make $x^{20}$ corresponds to a way to make $\$20$ using $\$1$ or $\$10$ coins, and we can indeed verify that there are three ways to make $\$20$ using $\$1$ or $\$10$ coins, which are $20 = 1 + 1 + \dots + 1 = 10 + 1 + \dots + 1 = 10 + 10$.

Functions like $f$ and $g$ are called \emph{generating functions} that are defined as the series $p(x) = \sum_{i=1}^{\infty} a_i x^i$. In general, let the coefficients $a_i$ describe the number of ways to construct $\$ i$ using coins of value $\$ a$. Now let another generating function be defined as $q(x) = \sum_{i=1}^{\infty} b_i x^i$, where the coefficients $b_i$ represent the number of ways to construct $\$ i$ using coins of value $\$ b$. If we multiply these functions together, we get another generating function $h$ where \[
  h(x) = p(x)q(x) = (a_0 + a_1x + a_2x^2 + \dots)(b_0 + b_1x + b_2x^2 + \dots) = \sum_{i=1}^{\infty} c_ix^{i}
.\] Now, each coefficient $c_i$ represents the number of ways to make $i$ using coins of value $\$ a$ or $\$ b$. We now return to the problem of making amounts using coins in Numberland, with our tools of generating functions.
\begin{theorem}
  The generating function for coins of value $a$ is \[
    g_a(x) = \frac{1}{1-x^a}
  .\] 
\end{theorem}
\begin{proof}
  Given coins of value $a$, we can only make amounts of value $ka$, where $k$ is an integer. Moreover, we can make these amounts in exactly one way, with $k$ coins. Thus the generating function, which can be rewritten as a geometric series, must be \[
    g_a(x) = x^{0a} + x^{a} + x^{2a} + \dots = \frac{1}{1-x^{a}}. \qedhere
  \] 
\end{proof}
\begin{theorem}
  For positive integers $n$, $C(n)$ is equal to the coefficient of $x^{n}$ in \[
    \prod_{a_i \in \NTen} \frac{1}{1-x^{a_i}}
  .\] 
\end{theorem}
\begin{proof}
  Recall that we can multiply generating functions together, and the resulting coefficients of each term $x^{n}$ represents the number of ways to form $n$. Thus we multiply the generating functions for all the numbers in the set \NTen{} to get the function \[
    \prod_{a_i\in \NTen} \frac{1}{1-x^{a_i}}
  ,\] where the coefficient of $x^n$ denotes the number of ways to form $n$, which is equivalent to $C(n)$.
\end{proof}
Unfortunately, it is unknown whether there exists a closed form expression to determine $C(n)$, which in the literature is referred to as a restricted partition function. Hardy and Ramanujan proved in 1918 \cite{hardyramanujan} that the partition function asymptotically approaches \[
  C(n) \sim \frac{a}{n}e^{b\sqrt{n}}
\] for constants $a$ and $b$.

We can, however, use computer software like Mathematica to calculate the coefficients in the generating function, and thus we can calculate $C(n)$ to get $C(1337) = 1448684$, which is the answer to Problem~\ref{p1}. We may graph the function $C(n)$, and although it looks exponential (Figure~\ref{f1}), the function is sub-exponential (as expected due to Hardy's asymptotic result). We can graph $C(n)$ using a logarithmic scale to showcase this, as seen in Figure~\ref{f2:log}.
\begin{figure}[htbp!]
  \centering
  \begin{tikzpicture}
    \pgfplotstableread{graph/something.dat}{\decreasing}
    \begin{axis}[height = 6cm, width = 0.9\columnwidth, xlabel={$n$}, ylabel={$C(n)$}, title = {$C(n)$ vs $n$}]
      \addplot[thick, red] table {\decreasing};
    \end{axis}
  \end{tikzpicture}
  \begin{tikzpicture}
    \pgfplotstableread{graph/something2.dat}{\decreasing}
    \begin{axis}[height = 6cm, width = 0.9\columnwidth, xlabel={$n$}, ylabel={$C(n)$}, title = {$C(n)$ vs $n$}]
      \addplot[thick, red] table {\decreasing};
    \end{axis}
  \end{tikzpicture}
  \caption{The number of ways $C(n)$ to make $n$ increases rapidly as $n$ increases.}
  \label{f1}
\end{figure}
\begin{figure}[htbp!]
  \centering
  \begin{tikzpicture}
    \pgfplotstableread{graph/something2.dat}{\decreasing}
    \begin{axis}[height = 6cm, width = 0.9\columnwidth, xlabel={$n$}, ylabel={$C(n)$}, title = {$C(n)$ vs $n$, log scale}, ymode = log]
      \addplot[thick, red] table {\decreasing};
    \end{axis}
  \end{tikzpicture}
  \caption{The number of ways, $C(n)$, grows sub-exponentially.}
  \label{f2:log}
\end{figure}
