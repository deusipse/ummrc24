\section{Problem 2}
\begin{definition}
  Let $\mathcal{L}_{10, 2}(N)$ be the minimum number of summands needed to construct $N$ using elements in \NTen.
\end{definition}
\begin{problem}\label{p2}
Determine $\mathcal{L}_{10, 2}(N)$ for the following values of $N$:
\begin{itemize}
  \item $1337$
  \item  $12345$
  \item  $190274876$
\end{itemize}
\end{problem}
For sake of convenience, we define $d_{n, i}$ to be the $i$th digit of $n$.
\begin{definition}
  For an integer $n$, the $i$th digit of  $n$, with the leftmost digit being the first, is denoted $d_{n, i}$.
\end{definition}
For example, $d_{1337, 4} = 7$.
\begin{theorem}
  For a positive integer $N$, $\mathcal{L}_{10, 2}(N) = \max\{d_{n, i}\}$.
\end{theorem}
\begin{proof}
  Firstly, we recognise that $\LTen(n)$ is equivalent to the minimum number of subtractions to make $0$ from $n$, using elements of \NTen. Intuitively, it is obvious that an optimal strategy would be to subtract $1$ from every non-zero digit in $n$, which decreases the largest digit by $1$ every time. This means that the total number of subtractions is simply equal to the largest digit of $n$. We can prove that this is the optimal strategy as part of a more general theorem.
\end{proof}
\begin{theorem}
  For a positive integer $N$, we have \[\mathcal{L}_{10, k}(N) = \ceil*{\frac{\max\{d_{n, i}\}}{k-1}}.\]
\end{theorem}
\begin{proof}
  Recall that in \NTen{}, the largest available digit is $k-1$. Again, the optimal strategy is to subtract $k-1$ from the largest digit in $n$ at every subtraction, until the largest digit is less than $k-1$, at which point it is obvious that it takes one more subtraction to reach zero. This gives us a total number of \[
    \ceil*{\frac{\max\{d_{n, i\}}}{k-1}}
  \] subtractions. We must now prove that it is impossible to do better than this. 

  Firstly, notice that if we subtract using numbers that are less than the largest digit every time, then the largest digit decreases by less than $k-1 $ each subtraction, meaning that we cannot beat the previous minimum number of subtractions. However, if we use subtractions that use digits larger than the largest digit in $n$, such as in $123-4$, we must be more careful.

  Consider the representation of a 4 digit number $n$ as concatenated digits $a|b|c|d$. Note that we can `group' digits to make an equivalent representation $(10a + b)|c|d$, in other words, we will allow `digits' that have two numbers.

  Now consider the subtraction $123-4$. The usual way we subtract numbers is that we `carry' a $1$ to the $3$ to do the subtraction $1|1|13 - 4 = 119$. Thus we see that the representation $a|b|c|d$ is equivalent to the representation $a|b|(c-1)|(d+10)$. This only occurs when we subtract using a digit that is greater than the largest digit in $n$. Notice that if we do not use any `alternate' representations, with the digits in their usual place, then every digit will be between $0$ and $9$, inclusive. This means that the \emph{maximum} number of subtractions must be \[
    \ceil*{\frac{9}{k-1}}
  .\] However, note that if we subtract using digits larger than the largest digit in  $n$, then we must have some pattern of  $(c-1)|(d+10)$ in the representation of  $n$. This means that the largest digit of $n$ now ranges from $10$ to $19$, inclusive. Importantly, this means that the \emph{minimum} number of subtractions is now \[
  \ceil*{\frac{10}{k-1}} \ge \ceil*{\frac{9}{k-1}}
  .\] Thus it always takes more (or the same) number of subtractions if we subtract using digits that are larger than the largest digit in $n$. Hence, the optimal strategy of subtracting using digits that are equal to the largest digit, is indeed the best possible strategy.
\end{proof}
Using this, we can easily determine the answers to Problem~\ref{p2}, which are in Table~\ref{t1}.
\begin{table}[H]
  \centering
  \begin{tabular}{lc} \toprule
    $n$ & $\LTen(n)$ \\ \midrule
    1337 & 7 \\
    12345 & 5 \\
    190274876 & 9 \\ \bottomrule
  \end{tabular}
  \caption{}
  \label{t1}
\end{table}
