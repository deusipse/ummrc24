\section{Problem 2}
\begin{definition}
  Let $\mathcal{L}_{10, 2}(n)$ be the minimum number of summands needed to construct $n$ using elements in \NTen.
\end{definition}
\begin{problem}\label{p2}
Determine $\mathcal{L}_{10, 2}(n)$ for the following values of $n$:
\begin{itemize}
  \item $1337$
  \item  $12345$
  \item  $190274876$
\end{itemize}
\end{problem}
For sake of convenience, we define $d_{n, i}$ to be the $i$th digit of $n$.
\begin{definition}
  For an integer $n$, the $i$th digit of  $n$, with the leftmost digit being the first, is denoted $d_{n, i}$.
\end{definition}
For example, $d_{1337, 4} = 7$.
\begin{theorem} \label{thm1}
  For a positive integer $N$, $\mathcal{L}_{10, 2}(n) = \max\{d_{n, i}\}$.
\end{theorem}
\begin{proof}
  Firstly, we recognise that $\LTen(n)$ is equivalent to the minimum number of subtractions to make $0$ from $n$, using elements of \NTen. Intuitively, it is obvious that an optimal strategy would be to subtract $1$ from every non-zero digit in $n$, which decreases the largest digit by $1$ every time. This means that the total number of subtractions is simply equal to the largest digit of $n$. We prove that this is the optimal strategy as part of a more general theorem in Theorem~\ref{thm:max}.
\end{proof}
