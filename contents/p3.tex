\section{Problem 3}
In general, let  $\mathcal{N}_{10, k}$ be the set of positive integers whose digits in base-10 are strictly less than $k$, where $k$ is a given positive integer. Let $\mathcal{L}_{10,k}(n)$ be the minimum number of summands needed to construct $n$ using elements in $\mathcal{N}_{10,k}$.
\begin{problem}
  Given the definitions, 
  \begin{itemize}
    \item Explore the relationship between $k$ and $\mathcal{L}_{10, k}(n)$.
    \item Define $L_{j, k}(n)$ and investigate the nature of $L_{j, k}(n)$ for positive integers $j, k$.
  \end{itemize}
\end{problem}
We expand on the ideas from Theorem~\ref{thm1} and prove a generalised formula for $\mathcal{L}_{10, k}(n).$
\begin{theorem} \label{thm:max}
  For a positive integers $n, k$, we have \[\mathcal{L}_{10, k}(n) = \ceil*{\frac{\max\{d_{n, i}\}}{k-1}}.\]
\end{theorem}
\begin{proof}
  Recall that in \NTen{}, the largest available digit is $k-1$. Again, the optimal strategy is to subtract $k-1$ from the largest digit in $n$ at every subtraction, until the largest digit is less than $k-1$, at which point it is obvious that it takes one more subtraction to reach zero. This gives us a total number of \[
    \ceil*{\frac{\max\{d_{n, i\}}}{k-1}}
  \] subtractions. We must now prove that it is impossible to do better than this. 

  Firstly, notice that if we subtract using numbers that are less than the largest digit every time, then the largest digit decreases by less than $k-1 $ each subtraction, meaning that we cannot beat the previous minimum number of subtractions. However, if we use subtractions that use digits larger than the largest digit in $n$, such as in $123-4$, we must be more careful.

  Consider the representation of a 4-digit number $n$ as concatenated digits $a|b|c|d$. Note that we can `group' digits to make an equivalent representation $(10a + b)|c|d$, in other words, we will allow `digits' that have two numbers.

  Now consider the subtraction $123-4$. The usual way we subtract numbers is that we `carry' a $1$ to the $3$ to do the subtraction $1|1|13 - 4 = 119$. Thus we see that the representation $a|b|c|d$ is equivalent to the representation $a|b|(c-1)|(d+10)$. This only occurs when we subtract using a digit that is greater than the largest digit in $n$. Notice that if we do not use any `alternate' representations, with the digits in their usual place, then every digit will be between $0$ and $9$, inclusive. This means that the \emph{maximum} number of subtractions must be \[
    \ceil*{\frac{9}{k-1}}
  .\] However, note that if we subtract using digits larger than the largest digit in  $n$, then we must have some pattern of  $(c-1)|(d+10)$ in the representation of  $n$. This means that the largest digit of $n$ now ranges from $10$ to $19$, inclusive. Importantly, this means that the \emph{minimum} number of subtractions is now \[
  \ceil*{\frac{10}{k-1}} \ge \ceil*{\frac{9}{k-1}}
  .\] Thus it always takes more (or the same) number of subtractions if we subtract using digits that are larger than the largest digit in $n$. Hence, the optimal strategy of subtracting using digits that are equal to the largest digit, is indeed the best possible strategy.
\end{proof}
Using this, we can easily determine the answers to Problem~\ref{p2}, which are in Table~\ref{t1}.
\begin{table}[H]
  \centering
  \begin{tabular}{lc} \toprule
    $n$ & $\LTen(n)$ \\ \midrule
    1337 & 7 \\
    12345 & 5 \\
    190274876 & 9 \\ \bottomrule
  \end{tabular}
  \caption{The minimum number of summands is equal to the largest digit.}
  \label{t1}
\end{table}
We now turn our attention to the second part of the problem. 
\begin{definition}
  For integers $j, k$, we define $\NN_{j, k}$ to be the set of numbers whose digits in base $j$ are less than $k$.
\end{definition}
\begin{definition}
  For integers $j, k$, we define $\LL_{j, k}(n)$ to be the minimum number of summands to construct $n$ using elements from $\NN_{j, k}$.
\end{definition}
This is not so different from the previously defined \NTen, apart from the fact that we now allow bases greater than $10$, such as allowing digits like $\mathrm{F}=15$ as in hexadecimal.

In fact, Theorem~\ref{thm:max} remains unchanged for $\LL_{j, k}$, that is, we have that \[
  \LL_{j, k}(n) = \ceil*{\frac{\max\{d_{n, i}\}}{k-1}}
\] for all positive integers $n, j, k$. The proof of this fact will follow identical reasoning to that of the previous theorem.
 \begin{theorem}
   For positive integers $n, j, k$, we have \[
  \LL_{j, k}(n) = \ceil*{\frac{\max\{d_{n, i}\}}{k-1}}.
\]
\end{theorem}
\begin{proof}
  We adapt our previous argument to work for digits of base $j$. Again, we consider the analogous problem of finding the minimum number of subtractions from $n$ to make $0$. 

  Like previously, the optimal strategy of subtracting $k-1$ from the largest digit until the largest digit is less than $k-1$ works, and the maximum number of subtractions using this method is  \[
    \ceil*{\frac{j-1}{k-1}}
  .\] The case where we subtract less than $k-1$ from the largest digit is trivially less optimal, and takes more subtractions, so we move on to the case where we subtract using digits larger than the largest digit.

  Consider a 4-digit number in base $j$, which we will denote $a|b|c|d$, where $a, b, c, d$ are digits in base $j$, and suppose $d$ is the largest digit. Since we are in base $j$, all the digits are less than $j$, where $j$ is an arbitrary integer. Now suppose we want to subtract $e$ where $e > d$. This is equivalent to the subtraction $a | b | c-1 | d+j - e$. Critically, notice that $d+j-e > d$, since  $j>e$. Thus the largest digit has now increased, meaning that we must make more subtractions. That is, a lower bound for the number of subtractions is now \[
    \ceil*{\frac{j}{k-1}}.
  \] Again, this shows that the optimal strategy indeed provides the minimal answer.
\end{proof}
